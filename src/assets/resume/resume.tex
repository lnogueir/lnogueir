    %----------------------------------------------------------------------------------------
%	PACKAGES AND OTHER DOCUMENT CONFIGURATIONS
%----------------------------------------------------------------------------------------

\documentclass[hidelinks]{resume} % Use the custom resume.cls style
\usepackage{fontawesome}
\usepackage{hyperref}
\usepackage{xcolor}
\usepackage{svg}
\usepackage[utf8]{inputenc}
\usepackage[T1]{fontenc}
\usepackage[default]{lato}


\protected\def\github{https://github.com/lnogueir}
\protected\def\linkedin{https://www.linkedin.com/in/lnogueir/}
\protected\def\website{https://lnogueir.me}
\protected\def\stackoverflow{https://stackoverflow.com/users/11348579/lnogueir}

 
\urlstyle{same}
\usepackage[left=0.7in,top=0.55in,right=0.7in,bottom=0.55in]{geometry} %

\usepackage{enumitem}

\newlist{bulletpoints}{enumerate}{1}
\setlist[bulletpoints,1]{
  label={$\bullet$},
  leftmargin=*,
  align=left,
  labelsep=2mm,
}

\newcommand{\tab}[1]{\hspace{\textwidth}\rlap{#1}}
\newcommand{\itab}[1]{\hspace{0em}\rlap{#1}}

\name{L\lowercase{ucas} N\lowercase{ogueira}\vspace{-.25cm}} % Your name

\address{3rd year student at the University of Waterloo}
\address{\href{\github}{\includesvg[width=14pt]{../images/githubLogo.svg}} \hspace{2px} \href{\linkedin}{\includesvg[width=14pt]{../images/linkedin.svg}} \hspace{2px} \href{\stackoverflow}{\includesvg[width=13pt]{../images/stack_overflow.svg}} \hspace{0.05px} Links \hspace{0.15px} | \hspace{0.15px} \href{\website}{lnogueir.me} \hspace{0.15px} | \hspace{0.15px} \href{mailto: lnogueir@uwaterloo.ca}{lnogueir@uwaterloo.ca} \hspace{0.15px} | \hspace{0.15px} +1 226 978-5884} \\  % Your phone number and email

\makeatletter
\newcommand{\globalcolor}[1]{%
  \color{#1}\global\let\default@color\current@color
}
\makeatother

\definecolor{niceblack}{HTML}{333333}

\AtBeginDocument{\globalcolor{niceblack}}


\begin{document}

%----------------------------------------------------------------------------------------
%	SKILLS SECTION
%----------------------------------------------------------------------------------------
\begin{rSubsection}{\textbf{Languages} | \normalfont{C++, Python, C, Go, JavaScript, Java, HTML, CSS}{}}{}{}

\end{rSubsection}
\vspace{-.2cm}
\begin{rSubsection}{\textbf{Tech} | \normalfont{WebRTC, Git, Docker, Node.js, React, MongoDB, AWS, Redis, Flask, nginx, Jenkins, MySQL}{}}{}{}

\end{rSubsection}

\vspace{-.20cm}

%----------------------------------------------------------------------------------------
%	EDUCATION SECTION
%----------------------------------------------------------------------------------------
\begin{rSection}{Education}
\vspace{-.1cm}
{\textbf{University of Waterloo | Bachelor of Computer Engineering}} \hfill {\textcolor{gray}{Sep 2018 - Apr 2023}} 
\\ {Specialization: Communication Systems and Signal Processing\\Relevant course load: Operating Systems, Data Structures \& Algorithms, Computer Networks}
\vspace{.1cm}
\end{rSection}

\vspace{-.20cm}
\begin{rSection}{Experience}
\vspace{-.1cm}
\begin{rSubsection}{\textbf{Software Engineering Intern} | \href{https://www.rossvideo.com/}{Ross Video} }{\textcolor{gray}{Remote | Jan 2021 - Present}}{}

    \begin{bulletpoints}
        \vspace{-.10cm}
        \item Working on the \href{https://www.rossvideo.com/products-services/infrastructure/softgear-software-based-signal-processing-platform/}{\underline{softGear}} Streaming Gateway to enable WebRTC streams in TV broadcast infrastructures.
        \vspace{-.13cm}
        \item Implemented native WebRTC client using \textbf{C++ chromium-based} API to stream multimedia content from Serial Digital Interface to media servers, achieving \textbf{ultra-low latency of \textasciitilde0.6s}.
        \vspace{-.13cm}
        \item Engineered \textbf{multi-threaded} media pipelines with \textbf{hardware acceleration} using \textbf{C} libraries to interact with a \textbf{GPU} and perform H264 video encoding/decoding efficiently, dropping CPU usage by \textbf{over 35\%}.
        \vspace{-.10cm}
    \end{bulletpoints}
\end{rSubsection}
\begin{rSubsection}{\textbf{Software Developer in Test, Coop} | \href{https://www.theweathernetwork.com/}{The Weather Network}}{\textcolor{gray}{Remote | May 2020 - Aug 2020}}{}
        \par
        \begin{bulletpoints}
            \vspace{-.10cm}
            \item Developed and \textbf{dockerized Go} micro-services to \textbf{concurrently} process weather files, contributing to deprecation of \textbf{C++} monolithic forecast engine serving \textbf{+4 million unique users}.
             \vspace{-.13cm}
            \item Performed \textbf{geospatial queries on MongoDB} to process coordinate messages from \textbf{AWS SQS} on \textbf{8GB} data sets of granular precision weather data leading to nearly \textbf{30\% performance upgrade}.
            \vspace{-.13cm}
             \item Used \textbf{Jenkins, Helm, and Kubernetes} to standardize \textbf{CI/CD deployment pipelines} across multiple engineering teams, improving infrastructure maintainability.
             \vspace{-.10cm}
        \end{bulletpoints}
\end{rSubsection}
\begin{rSubsection}{\textbf{Junior Web Developer} | \href{https://www.agf.com/ca/en/index.jsp}{AGF Investments}}{\textcolor{gray}{Toronto, ON | Sep 2019 - Dec 2019}}{}
        \par
        \begin{bulletpoints}
            \vspace{-.10cm}
            \item Shipped several \textbf{JSP} pages and \textbf{Java backend} server endpoints interacting with \textbf{SQL databases} to handle users/staff messaging requests and secure file uploads that \textbf{impacted over 60 000 users}.
            \vspace{-.13cm}
            \item Implemented \textbf{React} into \textbf{Java Spring} to migrate off legacy JSPs and jQuery Plugins which \textbf{reduced bundle size by approximately 15\%}.
             \vspace{-.10cm}
        \end{bulletpoints}
\end{rSubsection}
\begin{rSubsection}{\textbf{Teaching Assistant} | University Of Waterloo}{\textcolor{gray}{Waterloo, ON | Jan 2019 - Apr 2019}}{}

    \begin{bulletpoints}
        \vspace{-.10cm}
        \item Taught and coordinated \href{https://student.cs.uwaterloo.ca/~cs138/outline.shtml}{ \underline{CS138}}, a first-year Software Engineering course about \textbf{Data Structures} and \textbf{Object Oriented Programming} in \textbf{C++}.
        \vspace{-.13cm}
        \item Developed \textbf{Python} scripts to automate tasks such as assignments marking and course website updates.
        \vspace{-.10cm}
    \end{bulletpoints}   

\end{rSubsection}

\end{rSection}

\vspace{-.05cm}
%--------------------------------------------------------------------------------
%    Projects
%-----------------------------------------------------------------------------------------------


\begin{rSection}{Projects}
\vspace{-.1cm}
\begin{rSubsection}{\textbf{Liteboard.io} | Founder | \href{https://falauniversidades.com.br/projeto-gratuito-simplifica-o-acesso-as-aulas-on-line/}{\underline{Featured}} | +300 stars on \href{https://github.com/jeverd/lecture-experience}{\underline{Github} \faGithub}}{\textcolor{gray}{Apr 2020 - Oct 2020}}{}
        \par
        \begin{bulletpoints}
            \vspace{-.10cm}
            \item Built backend infrastructure, using \textbf{Janus SFU implementation}, capable of supporting \textbf{multistream conferences} with up to \textbf{50 participants} per room powered by \textbf{WebRTC, Redis, Node and Express}.
             \vspace{-.13cm}
            \item Implemented \textbf{WebSocket-based} chat rooms with support for image/document attachments to enable concurrent messaging system between students and lecturers in the platform.
            \vspace{-.25cm}
        \end{bulletpoints}
\end{rSubsection}

\begin{rSubsection}{{\textbf{Arithmetic Expression Tree Simulator}} |
\href{https://lnogueir.github.io/expression-tree-gen/}{\underline{Demo here } \faGithub}}{\textcolor{darkgray}{\textbf{Tech Stack} | JavaScript, HTML, CSS}}{}
\par
    \begin{bulletpoints}
    \vspace{-.10cm}
        \item A web app that, with animation, generates \textbf{binary trees} of arithmetic expressions with pure JavaScript by applying \textbf{Knuth's layout algorithm} to assign coordinate to nodes without collision of branches.
        \vspace{-.13cm}
        \item Applied \textbf{Dijkstra's Shunting-yard algorithm} to parse expression from infix to its postfix form to \textbf{asynchronously and recursively} create animated tree structure simulation.
        \vspace{-.25cm}
    \end{bulletpoints}
\end{rSubsection}

\end{rSection}






%----------------------------------------------------------------------------------------
%	TECHNICAL STRENGTHS SECTION
%----------------------------------------------------------------------------------------


\end{document}
